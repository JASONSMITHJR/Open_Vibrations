\documentclass[12pt,letter]{article}
\usepackage{mathptmx} % added for time new roman font
\usepackage[left=1in,right=1in,top=1in,bottom=1in]{geometry}
\usepackage[latin1]{inputenc}
\usepackage{amsmath}

\usepackage[textsize=tiny]{todonotes}

% defines all example enviorment
\usepackage[framemethod=tikz]{mdframed} % added for the box around examples
\newtheorem{ex}{Example}
\numberwithin{ex}{section} % allows for the use of example numbers that lign up with the section numbers
\newenvironment{example}{\begin{mdframed}[middlelinewidth=0.5mm]\begin{ex}\normalfont}{\end{ex}\end{mdframed}}

% defines all review enviorment
\usepackage[framemethod=tikz]{mdframed} % added for the box around examples
\newtheorem{re}{Review}
\numberwithin{re}{section} % allows for the use of example numbers that lign up with the section numbers
\newenvironment{review}{\begin{mdframed}[middlelinewidth=2mm,roundcorner=20pt]\begin{re}\normalfont}{\end{re}\end{mdframed}}

% defines the quotation enviorment 
\usepackage{xcolor}
\newcommand{\quotebox}[2]{\begin{center}\fcolorbox{white}{blue!15!gray!15}{\begin{minipage}{0.9\linewidth}\vspace{10pt}\center\begin{minipage}{0.8\linewidth}{\space\Huge``}{#1}{\Huge''}{\break\null\hfill} {\small #2}  \end{minipage}\medbreak\end{minipage}}\end{center}}

% defines the definition enviorment 
\newcommand{\definitionbox}[2]{\begin{center}\fcolorbox{white}{blue!15!gray!15}{\begin{minipage}{0.9\linewidth}\vspace{10pt}\center\begin{minipage}{0.8\linewidth} {{\textbf{Definition} - }{#1}: {#2}}\end{minipage}\medbreak\end{minipage}}\end{center}}

\usepackage{amsfonts}
\usepackage{amssymb}
\usepackage{graphicx}
\usepackage{float}
\usepackage{booktabs}
%\usepackage{parskip} % remove all the paragraph indents

\usepackage{setspace}
%\usepackage[colorlinks=true]{hyperref}
\usepackage{textcomp} 
\usepackage{multicol} 





%%%%%%%		define the symbols for positive directions		%%%%%%
\makeatletter													%%	
																%%					
\newcommand*\curveplus{% positive counterclockwise				%%
  \mathbin{\rotatebox[origin=c]{90}{$\m@th\curvearrowleft$}+}}	%%
																%%
\newcommand*\rightplus{% positive right							%%
  \mathpalette\@rightplus\relax}								%%
\newcommand*\@rightplus[1]{%									%%
  \mathbin{\vcenter{\hbox{$\m@th\overset{#1+}{\to}$}}}}			%%
																%%	
\newcommand*\upplus{% positive up								%%
  \mathbin{+\mathord\uparrow}}									%%
																%%			
\newcommand*\downplus{% positive down							%%		
  \mathbin{+\mathord\downarrow}}								%%
  																%%		
\newcommand*\downrightplus{% positive down and right			%%	
  \mathbin{+ \rotatebox[origin=c]{-30}{$\m@th\rightarrow$}}}	%%
\makeatother 													%%	
%%%%%%%%%%%%%%%%%%%%%%%%%%%%%%%%%%%%%%%%%%%%%%%%%%%%%%%%%%%%%%%%%%


\usepackage{mathtools}          %loads amsmath as well added for the piece wise function
\DeclarePairedDelimiter\Floor\lfloor\rfloor
\DeclarePairedDelimiter\Ceil\lceil\rceil

 
\newcounter{NumberInTable}
\newcommand{\LTNUM}{\stepcounter{NumberInTable}{(\theNumberInTable)}}

\newcommand{\Laplace}[1]{\ensuremath{\mathcal{L}{\left[#1\right]}}}
\newcommand{\InvLap}[1]{\ensuremath{\mathcal{L}^{-1}{\left[#1\right]}}}
\renewcommand{\textuparrow}{$\uparrow$}


\begin{document}



\setcounter{section}{7}	
\section{Experimental Vibrations}

Experimental vibration testing requires the practitioner to understand the basics of testing hardware and digital signal processing.

\subsection{Hardware for vibration testing}


\begin{figure}[H]
    \centering
    \includegraphics[]{../Figures/accelerometers.png}
    \caption{Integrated Electronics Piezo-Electric (IEPE) accelerometers, showing: (a) the cross section of a typical IEPE) accelerometer with key components annotated, and; (b) selection of IEPE accelerometers for various applications.}
    \label{fig:accelerometers}
\end{figure} 








\begin{table}[H]
\caption{Parameters for various accelerometers}
\resizebox{\textwidth}{!}{\begin{tabular}{@{}llllll@{}}
\toprule
\multicolumn{1}{c}{parameter} & \multicolumn{5}{c}{accelerometers} \\ \midrule
model number & PCB 393B31 & PCB  393B04 & PCB 352C67 & PCB 352A21 & PCB 352A92 \\
Sensitivity($\pm$ 10 \%) & 10.0 V/g & 1000 mV/g & 100 mV/g & 10 mV/g & 0.25 mV/g \\
Measurement Range & $\pm$ 0.5 g pk & $\pm$ 5 g pk & $\pm$ 50 g pk & $\pm$ 500 g pk & $\pm$ 20 kg pk \\
Frequency Range($\pm$ 5 \%) & 0.1 to 200 Hz & 0.06 to 450 Hz & 0.5 to 10 kHz & 1.0 to 10 kHz & 1.2 to 10 kHz \\
Resonant Frequency & \textgreater 700 Hz & \textgreater 2.5 kHz & \textgreater 35 kHz & \textgreater 50 kHz & \textgreater 100 kHz \\
Non-Linearity & $\le$1\% & $\ge$2.5 kHz & $\le$1\% & $\le$1\% &  \\
Transverse Sensitivity & $\le$5\% & $\le$5\% & $\le$5\% & $\le$5 \% &  \\ \bottomrule
\end{tabular}}
\end{table}


%Measurement Range & 0.5 g pk & \pm 5 g pk & \pm 50 g pk & \pm 100 g pk & \pm 500 g pk & \pm20,000 g pk \\
%Frequency Range(\pm 5 \%) & 0.1 to 200 Hz & 0.06 to 450 Hz & 0.5 to 10,000 Hz & 1 to 5,000 Hz & 1.0 to 10,000 Hz & 1.2 to 10,000 Hz \\


\begin{figure}[H]
    \centering
    \includegraphics[width=6.5in]{../Figures/IEPE.png}
    \caption{Integrated Electronics Piezo-Electric (IEPE)-based measurement system showing the: (a) simplified circuit schematic\protect\footnotemark[1]; and (b) IEPE data acquisition systems in various form factors.}
    \label{fig:IEPE}
\end{figure} 



\footnotetext[1]{``IEPE sensor connected to the input of an instrument'' by JanBurg  CC BY-SA 4.0}  




\subsection{Digital Signal Processing}





\begin{figure}[H]
    \centering
    \includegraphics[width=6.5in]{../Figures/signal_digitization.png}
    \caption{Digitization of two continuous time-series signals sampled at 5 S/s.}
    \label{fig:signal_digitization}
\end{figure}


The Nyquist?Shannon sampling theorem is a theorem in the field of signal processing that defines the sample rate that permits a discrete sequence of samples (i.e. discrete-time) to sample a continuous-time signal of a finite bandwidth. 

\begin{figure}[H]
    \centering
    \includegraphics[width=6.5in]{../Figures/aliasing.png}
    \caption{Aliasing of a 3 Hz signal that is sampled at 5 S/s.}
    \label{fig:aliasing}
\end{figure}

In signal processing, aliasing is an effect that causes different signals to become indistinguishable from each other.  In this way, the signals become an aliases of one another when sampled. Aliasing also accounts for the development of distortion or artifact in a reconstructed signal when compared to the original continuous signal.


\begin{figure}[H]
    \centering
    \includegraphics[width=6.5in]{../Figures/spectrogram.png}
    \caption{Spectrogram of a 0-10 Hz chirp signal.}
    \label{fig:spectrogram}
\end{figure}

Some of the key parameters in a spectrogram include:

\begin{itemize}
\item window
\item segment length
\item overlap
\end{itemize}

\end{document}














